\documentclass[11pt]{book}
\usepackage[utf8]{inputenc}
\usepackage[T1]{fontenc}
\usepackage{geometry}
\usepackage{graphicx}
\usepackage{uarial}
\usepackage[spanish]{babel}
% ubica las figuras y tablas a X cm del margen tanto derecho como izquierdo, pone el label de la figura en negritas y tanto el texto como el label en una fuente mas pequeña (del mimso tamaño quie los pies de pagina). 
\usepackage[margin=1.5cm,font=footnotesize,labelfont=bf]{caption}
\makeindex

%\title{\textbf{Caminos geodésicos en el universo anisótropo de Gödel }}
%\author{Aguirre Astrain Angelica}
%\date{\today}
\begin{document}
%\vspace{5em}
%\maketitle

%%% CARÁTULA
\newdimen\MSup
\newdimen\MIzq

%% Valores ajustables
\MSup=50pt   
\MIzq=50pt  



\newenvironment{changemargin}[1]{%
\begin{list}{}{%
\setlength{\leftmargin}{#1}%
\setlength{\topmargin}{-100pt}
\setlength{\parsep}{\parskip}%
}\item[]
}{\end{list}}

\addtolength{\topmargin}{-\MSup}

\begin{changemargin}{-\MIzq}
\thispagestyle{empty}
\begin{minipage}[c][1pt][t]{0.2\paperwidth}
\begin{center}

\includegraphics [width=100 pt ]{esc}\\
\vskip 20pt
\hskip -10pt
\linethickness{1.6pt} 
\line(0,1){520}
\linethickness{0.9pt} 
\line(0,1){500}
\linethickness{1.6pt} 
\line(0,1){520}
\end{center}
\end{minipage}
\hskip 20 pt
\begin{minipage}[c][1pt][t]{0.6\paperwidth}
\begin{center}
\vskip 30pt
{\LARGE \scshape Universidad Veracruzana}
\linethickness{1.6pt} 
\line(1,0){350}\\
\linethickness{.9pt} 
\line(1,0){313}
\vskip 10pt


{\Large \scshape Facultad de F\'isica }


\vskip 60pt


{\LARGE \textbf{ Caminos geodésicos en el universo anisótropo de Gödel
}}\\

\vskip 70pt

{\Large Trabajo recepcional en la modalidad de:}\\

\vskip 12pt

\textbf{\LARGE TESIS}


\vskip 12 pt
{\Large que como requisito pacial para obtener el t\'itulo de:}
\vskip 12pt

\textbf{\LARGE Licenciado(a) en F\'isica}
\end{center}

\vskip 12pt

\begin{center}
{\Large {P\ R\ E\ S\ E\ N\ T\ A} }
\end{center}

\vskip 12pt

\begin{center}
\textbf{\LARGE  ANGELICA AGUIRRE ASTRAIN}
\end{center}

\vskip 70pt

\begin{center}
{\large ASESORES:}\\
\vskip 20 pt
%Efraín?
{\large  DRA. HILDA NOEMI NÚÑEZ YÉPEZ }\\

{\large  DR. JUAN EFRAÍN ROJAS MARCIAL }\\
\vskip 75 pt


Xalapa de Enr\'iquez, Veracruz\hfill Mes  a\~no


\end{center}
\end{minipage}
\end{changemargin}

\newpage

\addtolength{\topmargin}{\MSup}
%%%



%Para empezar la numeración en números romanos
\pagenumbering{Roman}
\tableofcontents
\cleardoublepage

%Propósitos:-Dar a conocer el propósito de Gödel al proponer un modelo de universo que desafía nuestra noción intuitiva del tiempo.

% NOTAS
% -palabra- : revisar si es la correcta para expresar algo

%Para empezar la numeración en arábico
\pagenumbering{arabic}

\chapter{Introducción}
%\section{Introducción}

%Comienzo, la ciencia como un modelo que intenta reproducir la ''realidad'' que percibimos. Muchas veces esta idea se deja de lado y la ilusión de que las teorías son la realidad toma su lugar y no nos deja ver que quizá hay algo más allá de lo que podemos o podremos describir y predecir. (\cite{inverno})

Según la defición de la Real Academia Española, la ciencia es el ``conjunto de conocimientos obtenidos mediante la observación y el razonamiento, sistemáticamente estructurados y de los que se deducen principios y leyes generales con capacidad predictiva y comprobables experimentalmente'' ---y se puede encontrar una clasificación de la ciencias, entre las que se encuentran las ciencias sociales, naturales, exactas, etcétera. % REVISAR la clasificación de las ciencias
A diferencia de las demás, al éxito que las ciencias exactas han logrado es por su capacidad de describir mucho de lo que percibimos como \emph{realidad} y más allá de eso predecir  eventos futuros al conocer ciertas condiciones iniciales. % (en los casos ideales/ mecánica cuántica)
 Pero no debemos olvidar que, para nuestro caso, toda ley física propuesta no es más que un intento del hombre para describir lo que experimenta a traves de los sentidos y observaciones su entorno. Es en general, una ``muy buena'' aproximación a la realidad.

Por ejemplo, la ley que regía la gravitación propuesta en el siglo XVII tuvo que ser reemplazada por que implicaba la existencia de una fuerza que afectaba de forma instantánea a dos objetos %y que su magnitud solo dependía de el inverso del cuadrado de la distancia que los separa
y aunque su capacidad predictiva prevalece hasta nuestros días, se sabía ---o, al menos, Newton lo intuía---
que solo era una teoría provisional hasta que una mejor apareciera, cosa que sucedió hasta casi 400 años después. Fue a principios del siglo XX que un científico  hizo algo impensable, desafió la noción de la gravitación newtoniana. La gravedad pasó de ser una fuerza a una manifestación de la estructura del espacio-tiempo en el que se encuentra un objeto, espacio-tiempo que se verá afectado todos los cuerpos en él, deformándose como consecuencia y, como resultado,  el objeto modificará su trayectoria.
Esta reformulación fue posible porque Einstein sabía que la teoría newtoniana de la gravedad no era más que  una descripción de la realidad,  no la realidad misma.

 \section{Gödel y los teoremas de incompletitud }
%breve biografía

%Kurt Friedrich Gödel, el menor de dos hijos nacido el 28 de abril de 1906 en Austria? hijo de Rudolf y Marianne Gödel 

% Importancia del teorema de incompletitud en matemáticas y su aplicación a otros campos, la computación y las leyes. y en la siguiente sección resaltar la importancia de estos teoremas en la física.

%Hasta (fecha/siglo) las matemáticas se basaban/ eran una estructura de forma:

A finales del siglo XIX los matemáticos se vieron en apuros al enfrentarse con inconsistencias y paradojas en sus planteamientos, fueron principalmente tres aconteciemientos los que dieron lugar a -la necesidad de cambiar de rumbo-
% REVISAR

$\circ$(?) en --- con el desconcierto de no poder expresar $\sqrt{2}$ como la suma de fracciones. Esto llevó a la noción de los números irracionales.

$\circ$ en --- Cantor -demostró- que $\infty$ no es un número es el conjunto de una cantidad muy grande de números -reales(?)- y que existen infinitos mas grandes que otros. % el infinito de los números reales es más grande que el de los números -naturales-
Esta nueva concepción que para muchos matemáticos resultaba como una barbaridad, para otros era una oportunidad/el comienzo de una nueva era con grandes posibilidades. 

$\circ$en --- la paradoja de Rusell, el nacimiento de la -teoría de grupos/conjuntos-

%Hilbert se preguntó si un sistema lógico simbólico podría ser completo y consistente

David Hilbert, %conocido seguidor del positivismo (?), en --- 
basado en sus fuertes creencias positivistas propuso que una analogía de las matemáticas con la teoría física sería que, mientras la física se encarga de describir los fenómenos naturales con base en ciertas leyes puestas a prueba con experimentos cognitivos, la matemática %(aritmética ?) 
debía estar basada en una serie de axiomas y procedimientos rigurosos de comprobación. 
Hilbert apuntaba a generar un formalismo matemático que nos permitiera conocer todas las verdades matemáticas a partir de demostraciones.


Sin embargo el joven Gödel de tan solo 25 años,  %(como parte de su tesis doctoral de ---) 
en su intento de ``ayudar'' a Hilbert termino truncando su propósito, lo que marcaría un hito en la historia de las matemáticas y su desarrollo. En 1931 Gödel presenta sus teoremas de incompletitud que cuestionan todo intento de hacer un sistema formal completo y consistente basado en axiomas. % que, de acuerdo con el primer teorema, describa todas las verdades de la aritmética de los números naturales y que se pueda demostrar su consistencia a partir de solo los axiomas propuestos en él (segundo teorema).
%citar los teoremas

Tras varios años viviendo en Estados Unidos Gödel solicitó la residencia Americana, y fueron sus amigos Einstein y Morgenstern los testigos que lo acompañaban. % además con el propósito de no dejarlo -jugar/jeopardizar- con los que tenían el poder de  negarle la ciudadanía. 
La historia cuenta que durante esta -sesión- Gödel comentó de la posibilidad de que el gobierno de los Estados Unidos se volviera una dictadura.../que por más cuidadosos que fueran al formular las leyes, aplicando uno de sus teoremas de completitud, nunca podrán abarcar todas las posibilidades que pudieran acontecer y necesitar el juicio legal. Por suerte el que precedía la sesión era -conocido/amigo- de Einstein(?) y él y Morgenstern pudieron detenerlo antes de que pudiera empezar un debate que pondría en juego su ciudadanía.

\section{Una curiosa amistad}
% Relación de Gödel y Einstein (a world without time)
% CORROBORAR FECHAS

Marianne Gödel dió a luz (a su segundo hijo?) solo un año después de la primera gran obra de Albert Einstein, pero la diferencia de edad de casi 30 años no fue un impedimento para el desarrollo de lo que sería una gran amistad.

En 1930 se fundó el Instituto para Estudios Avanzados en Princeton, New Jersey con el propósito de ser una institución especializada en Física y Matemáticas %(?)
y contar con una plantilla de los mejores exponentes de la época en éstas áreas. Desde su fundación a albergado a personajes como ---, Jon von Neumann, Einstein y por supuesto Gödel. % A la fecha existen -- instituciones de este tipo especializadas en diferentes áreas de la ciencia en todo el mundo.

Fue en 1933 que Albert Einstein se unió al IAS (por sus siglas en inglés) como -profesor en donde no se le exigía impartir clases- 

Kurt Gödel estuvo en tres ocasiones %sabáticos
en esta institución y a partir de 1940 trabajó de forma permanente hasta su muerte en 1978. En esta época fue donde conoció a su compatriota Einstein donde surgió una amistad conocida por todos en el instituto debido a sus característicos paseos a pie donde platicaban por horas.% aqunque los temas específicos se desconocen 
Se sabe que una razón -del éxito de la amistad- fue que el joven Gödel no temía expresar sus ideas y puntos de vista...

Ambos compartían intereses que los hacían relacionarse, Gödel disfrutaba de la música clásica %como: ...
el teatro, la opera(?), mientras Einstein era un violinista habilidoso. Además ambos tenían inclinaciones filosóficas, un punto clave para...

Con la teoría relativista Einstein se había cuestionado sobre la naturaleza de lo que todos daban por sentado, el tiempo y el espacio son absolutos. Esta noción de alguna manera intuitiva, por el orden de magnitud de la velocidad a la que nos movemos,
 y por lo tanto no tan fácil de desechar 
 
 %Kurt y Kant 



\chapter{Breve repaso de la teoría general de la relatividad }



%En 1905 Einstein propuso su teoría de la relatividad especial, pero fue Minkowski quien hizo la representación matemática que daría lugar al siguiente gran paso, la formulación de la relatividad general.

Las transformaciones de Galileo (\ref{galileo}) gobernaban la mecánica newtoniana en donde el espacio era considerado como absoluto y estático, de la misma forma que el tiempo.

\begin{equation}
 x=x'+vt \qquad y=y' \qquad z=z' \qquad t=t' ,
 \label{galileo}
\end{equation}

Pero las ecuaciones de Maxwell que describen el electromagnetismo,no son invariantes bajo estas transformaciones. Como estas ecuaciones eran las indicadas para describir los fenómenos electromagnéticos, las transformaciones indicadas en este caso fueron las siguientes:

\begin{equation}
x'=\frac{x - v_x t}{\sqrt{1-\frac{v^2}{c^2}}} \qquad t'=\frac{t-xv_x}{\sqrt{1-\frac{v^2}{c^2}}}
\label{lorentz}
\end{equation}


estas son las conocidas transformaciones propuestas por Hendrick Antoon Lorentz a principios del siglo XX %pero fue Poincaré quien les dió la forma que usamos hoy en día
, con las que las ecuaciones de Maxwell permanecen invariantes y para que para cualquiera dos observadores la velocidad de la luz permanece constante, que era interpretado como a la existencia de un marco de referencia privilegiado, conocido como éter, un/una ''algo/sustancia'' invisible e inperceptible con respecto al que la luz se mueve.

Esta idea de un algo imperceptible y que por lo tanto que no se puede medir va en contra a las ideas positivistas(?), algo que no es comprobable experimentalmente  no debería ser parte de la teoría fundamental.
 
No fue hasta que Einsten propuso una explicación a las transformaciones de Lorentz, conclusiones que desafiaban la intuición dado que  la posición, la velocidad, incluso el tamaño de un objeto y el tiempo que le tomaba realizar una acción dependen de la velocidad relativa de este con respecto al observador. Además, en esta teoría se tiene una cota fija que limita la velocidad a la que puede viajar una partícula, la velocidad de la luz $c$ . 

Si por ejemplo, al observar una partícula material te percatas de que comienza a moverse a velocidades cercanas a la velocidad de la luz, lo que verás es que ocurre un cambio en la partícula, comienza a aplastarse/achatarse -en dirección de su movimiento- y además su masa aumenta, esto es debido a la presencia de un factor gama 

\begin{equation}
\gamma=\frac{1}{\sqrt{1-\frac{v^2}{c^2}}} ,
\label{fgamma}
\end{equation}


%%% Diagrama que muestre la acción de gamma en la longitud(tamaño) y tiempo que le tomó al cuerpo hacer el recorrido.

en las relaciones de transformación tanto de longitud como de masa. Más increíble aún puede resultar %la siguiente situación: el observador A se encuentra en una estación de lanzamiento en la Tierra y el observador B en una nave que esta a punto de despegar de esa estación para hacer un viaje a Júpiter, ambos llevan consigo un reloj que sincronizan de tal manera que ambos empiecen a correr en el momento del despegue. En cuanto el observador B comienza su viaje, se mueve a una velicidad de $0.98c$ . Cuando el observador B aterriza en Júpiter, inmediátamente manda una señal a la Tierra para confirmar su llegada egún el reloj de A
que si un observador $A$ mide el tiempo que le toma hacer cierto viaje a un observador $B$ que se mueve a velocidades cercanas a la velocidad de la luz, la conclusión será que el tiempo que mide $A$ desde su sistema de referencia es mayor al que mide $B$, es decir, en el sistema de referencia de $B$ el viaje fue más corto. Pero hay que recordar que, no por cada uno tener una medición  de tiempo diferente la medicón de uno de ellos es válida mientras que la del otro observador no lo es, al contrario ambos obsevadores están en lo correcto. 

Las transformaciones de Lorentz -restringen- de forma explícita el movimiento de una partícula, que aparezca el factor gamma en la relación de masa implica que mientras más rápido una partícula se mueva, mayor energía necesita.

\section{Métrica}

El elemento de línea expresa el cuadrado del intervalo infinitesimal entre dos eventos $x^a$ y $x^a+dx^a$, escrito en su forma tensorial como

\begin{equation}
ds^2=g_{ab}dx^adx^b ,
\label{linea}
\end{equation}
donde $g_{ab}$ representan las componentes del tensor métrico, que es una matriz -simétrica- de $4 \times 4$ .

En la relatividad especial de Einstein se analiza el espacio tiempo plano dado por la métrica de Minkowski 
\begin{equation}
ds^2=-c^2dt^2+dx^2+dy^2+dz^2 ,
\label{minkowski}
\end{equation}


En este espacio se presenta el movimiento de una partícula libre de cualquier -perturbación- gravitacional y se introducen los conceptos de: línea de mundo, tiempo propio, conode luz,.... Las líneas de mundo son la representación en cuatro dimensiones de la trayectoria de una partícula, además de -mapear las tres coordenadas espaciales, las evoluciona en el tiempo-. % Diagrama de la línea de mundo
% EXPLICAR cómo obtener el cono de luz
Los conos de luz limitan las -trayectorias causales- que una partícula material puede seguir. Como se sabe que lo más rápido que puede viajar una partícula es a la velocidad de la luz, y esto solo los fotones que carecen de masa, para cualquier otra partícula que solo puede alcanzar velocidades menores sus trayectorias posibles se deben encontrar dentro del cono de luz. 

Para facilidad en la visualización se elige representar el espacio tiempo tetradimensional como el conjunto de hipersuperfiecies en donde en cada una se encuentran ``contenidas'' las tres dimensiones espaciales para un tiempo dado. Ahora, se ubica una partícula en el punto $P$ de la hipersuperficie para el tiempo $t=t'$ y con ella su respectivo cono que representa a que puntos para un tiempo futuro esta partícula tiene acceso así como tambíen los posibles puntos de los que pudo haber venido desde su trayectoria pasada hasta llegar a $P$.

Cualquier punto $P'$ dentro del cono de luz futuro es un punto alcanzable para esta partícula ya que solo requiere velocidades menores a $c$ , conforme este punto se acerque más al cono, requiere velocidades mayores para llegar, de la misma forma, si el punto $P''$ se encuentra dentro del cono pasado el punto $P$ le es perfectamente alcanzable.
Ahora, si la partícula en el punto $P$ quisiera llegar al punto $Q$ solo podrá alcanzarlo si estamos hablando de una partícula de luz, por que requiere moverse a $c$. Un último caso es cuando la partícula quiere alcanzar el punto $R$ que se encuentra fuera de su cono de luz pero en la misma hipersuperficie en un tiempo constante
%%% REVISAR: si es necesario comentar dos veces el significado de la imágen, revisar artículos (rindler, visualisation of GU.)
\begin{figure}[h!]
\centering
\includegraphics[height=2in]{conodeluz}
\caption{ La línea de mundo de una partícula en el punto $P$ se -debe- encontrar en todo punto dentro del cono de luz de la misma, y tiene -permitido- llegar a $P$ desde un punto $P''$ en el pasado y a partir de $P$ alcanzar el punto $P'$ en el futuro siempre moviéndose a velocidades menores a $c$. solo un fotón puede ir de $P$ a  cualquier punto $Q$ en la superficie del cono. Para alcanzar un punto $R$ fuera del cono la partícula necesita viajar a velocidades mayores a $c$}
\label{conodeluz}
\end{figure}

A partir de la métrica se pueden derivar estas restricciones
\begin{equation} 
ds^2=0 \qquad \to \qquad c^2dt^2=dx^2+dy^2+dz^2 \qquad partícula~ de~ luz 
\end{equation}

\begin{equation}
ds^2<0 \qquad \to \qquad c^2dt^2>dx^2+dy^2+dz^2 \qquad partícula~material
\end{equation}

\begin{equation}
ds^2>0 \qquad \to \qquad c^2dt^2<dx^2+dy^2+dz^2 \qquad partícula~con~velocidad~mayor ~a~c
\end{equation}

Solo a un plano se le puede asignar una métrica del tipo Minkowsky, mientras que para cualquier otro espacio curvo el elemento de línea que lo describe

\section{Geodésicas}

Para cualquier espacio se puede encontrar una curva específica que conecte dos puntos y que esta curva sea la más corta posible. Para el espacio plano esta curva es siempre una línea recta, pero en otros espacios estas curvas pueden no ser tan triviales.

La ecuación \ref{geodesicas} es la ecuación de las geodésicas 
\begin{equation}
\frac{d^2x^\alpha}{d\tau^2}+\Gamma^\alpha_{\beta \gamma} \frac{dx^\beta}{d\tau}\frac{dx^\gamma}{d\tau}=0,
\label{geodesicas}
\end{equation}

donde toda la información sobre la curvatura del espacio está contenida en los símbolos de Christoffel que están en función de las primeras derivadas de la métrica
\begin{equation}
\Gamma^\alpha_{\beta \gamma}=\frac{1}{2}g^{\alpha \beta} \left( \frac{\partial g_{\beta \sigma}}{\partial x^\alpha} + \frac{\partial g_{\gamma \\beta}}{\partial x^\beta} - \frac{\partial g_{\alpha \beta}}{\partial x^\sigma} \right),
\label{christoffel}
\end{equation}



\chapter{Universo de Gödel}

%%% Completar esta sección muy básicamente y ver si                                                                                                                                                                                                                                                                                                                                                                                                                                                                                                                                                                                                                                                                                                                                                                                                                                                                                                                                                                                                                                                                                                                                                                                                                                                                                                                                                                                                                                                                                                                                                                                                                                                                                                                                                                                                                                                                                                                                                                                                                                                                                                                                                                                                                                                                                                                                                                                                                                                                                                                                                                                                                             es necesaria.
%\section{Gravitación Newtoniana}
% Gravitación Newtoniana (MTW(?), GR Price )

\section{Derivación del modelo de universo de Gödel}
% Derivación del GU (Rindler)

%Al parecer Gödel tenía claro que un universo en rotación excluia la posibilidad de asignarle un tiempo cosmológico a un modelo de este tipo, fue de ahí que partió su idea de crear un modelo y en analogía con nuestro universo también quería que fuera homogeneo. %Una gran diferencia con los modelos de Friedman en los que el universo es homogéneo e isótropo.
%A partir de estas restricciones Gödel empezó a trabajar en lo que, a pesar de no ser el primer modelo que proponía un universo rotante si profundizaba en las implicaciones que estas características conllevan.

%Ahora se hará una breve derivación de este modelo de universo siguendo el -orden lógico- que utilizó Gödel, pero con un procedimiento diferente.

En 1939 se publicó el primer volumen de \emph{The Library of Living Philosophers}, creación de Paul A. Schilpp quien fue el editor hasta los años  ochentas. El propósito de este conjunto de libros es básicamente presentar la autobiografía intelectual del personaje a quien va dedicado el volumen, además de una serie de -artículos- de sus contemporáneos en los que presentan cuestionamientos a distintos trabajos y para los que resiven respuesta del -homenajeado-, publicado todo en el mismo volumen. La idea es tener un compendio de esta interacción donde se espera obtener el mayor provecho al -sacar a la luz dudas o comentarios a los que el autor al responder ayuden a aclarar y ampliar el conocimiento-.

Como contribución al séptimo volumen dedicado a su gran amigo Albert Einstein, Kurt Gödel se propuso hacer un escrito titulado ''Some remarks about the relation between the theory of relativity and Kant''

un modelo de universo en rotación consistente con la relatividad general.
Además, al parecer Gödel tenía claro que un universo en rotación excluía la posibilidad de asignarle un tiempo cosmológico a un modelo de este tipo, con esto en mente estaría 

 fue de ahí que partió su idea de crear un modelo y en analogía con nuestro universo también quería que fuera homogeneo.



\section{Características del modelo}
% Características (9-Gödel's)
%Demostraciones

(1) \emph{Solución estacionaria y espacialmente homogénea}  

$\circ $ ~ Por construcción,  procedimiento de Rindler

$\circ$~ Análisis de: On Gödel and the ideality of time

(2)\emph{Grupo uniparamétrico de transformación; cualquiera dos líneas de universo son equidistantes}

$\circ$~ Parámetro de desviación geodésica ($\xi$)?

(3)\emph{S tiene simetría rotacional}

$\circ$~ Gödel's demostration

$\circ$~ Cambio de coordenadas

(4)\emph{A todos los vectores temporales y nulos se les puede asignar una dirección temporal}

$\circ$~ ...

(5)\emph{No se puede asignar una coordenada de tiempo $t$ a cada punto del espacio tiempo de tal forma que $t$ siempre aumente}

$\circ$~Revisar demostración de Gödel

$\circ$~No se puede asignar un world time al modelo por la anisotropía- No se puede foliar el modelo- d'inverno/schwarzschild

(6)\emph{Toda línea de universo de materia es abierta y de longitud infinita que nunca se acerca a ninguno de sus puntos anteriores, pero también existen curvas de materia cerradas}

$\circ$ ~ ...

(7)\emph{No existen hipersuperficies espaciales que intersecten cada linea de universo de materia}

$\circ$~Revisar demostración de Gödel

$\circ$~ No se puede foliar/ Esta enlazado con (6)

(8)\emph{No existe un tiempo absoluto}

$\circ$~Revisar demostración de Gödel

$\circ$~Proposición de la world time

(9)\emph{La materia gira relativa al compás de inercial velocidad angular $2(\pi\kappa\rho)^\frac{1}{2}$}

$\circ$~Revisar demostración de Gödel

$\circ$~Por construcción de Rindler

% Diferencia con nuestro universo/ por que esta métrica no describe nuestro universo, un universo en expansión pero no rotando(wwt, la búsqueda de la rotación de todas las galaxias en una dirección específica).


\vspace{3em}


\emph{FALTA UBICAR EN EL TEXTO}
% FALTA UBICAR
% Alguna sección en la que se discuta brevemente la ''definición'' de tiempo y los problemas que presenta.

Noción de tiempo de Kant.

Si el tiempo consiste en la sucesión de capas infinitas de ''ahora'' que comienzan a existir una después de otra. Tomando en cuenta que la noción de simultaneidad depende de cada observador esto implicaría que cada observador tiene su propio tiempo es decir, un conjunto de ''ahoras'' \cite{remark}, en el que ninguno puede decir que el suyo es el dominante, definitivo o absoluto.

Ahora bien, si en el modelo de universo propuesto por Kurt Gödel, tal como él remarcaba, debido a las características impuestas no se le puede asignar un tiempo cosmológico válido para todo (?) observador a forma de analogía con el tiempo absoluto newtoniano, y como existen ciertas trayectorias que una partícula puede seguir para viajar en el tiempo en este modelo, donde esto implica que de alguna forma tiene acceso las capas pasadas de su tiempo, la noción de tiempo impuesta pierde su sentido.
Lo que Gödel proponía era, como su modelo era una solución a las ecuaciones de Einstein que desafiaba la idea ''intuitiva'' de un tiempo que simpre va hacia adelante, había solo dos opciones, que la relatividad general estuviera errada(?) o que nuestra noción de tiempo no era correcta, él optó por la segunda.
% 
Sobre esto, una de las cosas que molestaba a Einstein sobre la teoría newtoniana es la existencia de una fuerza que actuaba de forma instantánea afectando a dos cuerpos de modo que esa fuerza es siempre atractiva. E encontró que lo más rápido que una señal puede viajar es precisamente la velocidad de la luz, por lo que ni la gravedad puede superar esta velocidad, en este sentido no puede ser instantánea. 

Como Rindler lo menciona \cite{rindler}, Gödel ''..incitó a los astrónomos a buscar evidencia de rotación (en las galaxias), y a los filósofos a repensar las ideas sobre el tiempo.'' 
%

% Propósito del universo de Gödel (UG)
%El universo propuesto por Gödel, debido a la distribución de materia que presenta, se encuentra deformado de tal forma que una nave viajando a una velocidad suficientemente grande análisis de Gödel de la velocidad y combustible necesarios para el viaje  \cite{remark} puede viajar en una trayectoria hacia el futuro para después devolverlo a algún punto en su pasado.
%En el espacio vacío existe una completa equivalencia entre obsevadores que se mueven a velocidades uniformes diferentes
%La presencia de materia deforma el espacio y da a los observadores percepciones naturales o distorcionadas más allá del hecho de que sean equivalentes y sigan las mismas leyes de movimiento. Dentro de las soluciones de las ecuaciones de campo conocidas, bien comportadas (?), todos de los obsevadores pueden encajarse en un \emph{world time} que se puede tomar como el tiempo verdadero y de cierta forma volvemos a la antigua concepción newtoniana de tiempo y transcurre de una forma objetiva, tal como Gödel lo plantea, para todos estos observadores.
%Explicar world time
%¿Que sucede entonces para los espacios, como los universos rotatorios, para los que no se puede definir un world time o tiempo cosmológico absoluto? Esta pregunta la plantea Gödel, y resulta que para estas situaciones estos espacios además de no contener esta definición de tiempo poseen otras características que podrían permitir a un observador viajando a una velocidad considerable, en comparación con la velocidad de la luz, llegar al futuro y al continuar con el viaje regresar a algún punto en su pasado encontrandose con su yo de ese entonces.

%Si estamos de acuerdo en que la relatividad general es la ley suprema que rige las leyes físicas y que ha pasado las pruebas 

% La RG nos dice en que tipo de modelos se van a cumplir las mismas leyes físcas que se cumplen en nuestro universo(????)

Su argumento fué que la relatividad general pone una serie de restricciones a lo que se puede ''generar'' como una realidad, cualquier modelo que soluciones las ecuaciones en principio, su existencia es podible y se van a cumplir las leyes físicas que conocemos. Entonces si en este modelo, homogeneo y rotando que resulta ser una solución mucho más general a las ecuaciones que el modelo que describe el universo en el que vivimos, no hay necesidad de definir un tiempo cosmológico, la definición de este tiempo podría estar demás en el nuestro por el simple hecho de que nuestra noción del cambio en el tiempo nos obliga a proponer uno.

\chapter{Geodésicas en el modelo UG}

%POSIBLE sección explicando la importacia de conocer las geodésicas en un modelo cosmológico. Las geodésicas descrien las trayectorias que partículas en ''caída libre'', partículas libres que solo se ven influenciadas por un campo gravitacional.

%BUSCAR otra forma de calcular geodésicas, Rindler


\section{Ecuaciones de las geodésicas}

%REVISAR que esté bien ubicado, creo que las especificaciones deben ir al final/Conclusiones.
%La propuesta de Gödel sobre la existencia de ciertas curvas que una partícula de materia podría seguir para volver a algún punto en su pasado, en su artículo \cite{godel} solo presenta la posibilidad más no da una curva definida por otro lado, en su contribución al séptimo volumen de \emph{The Library of Living Philosophers} \cite{remark}comparte algunas especificaciones para la nave que realizará el viaje, donde la velocidad de la nave debe ser de al menos $1/\sqrt{2}$ la velocidad de la luz además, la cantidad de combustible necesaria para un viaje de $t$ tiempo de duración debe ser $10^{22}/t^2$ veces el peso de la nave. %revisar valor de t? (t<<10'')

El elemento de línea de este modelo asumiendo/tomando unidades geométricas $c=1$ es:
\begin{equation}
ds^2=a^2(dx_0^2-dx_1^2+(\exp(2x_1)/2)dx_2^2-dx_3^2+2\exp(x_1)dx_0dx_2),
\label{elinea}
\end{equation}

y la forma matricial de esta métrica
\begin{equation}
g_{_{GU}}=\left(
\begin{array}{cccc}
1 & 0 & \exp(x_1) & 0 \\
0 & -1 & 0 & 0 \\
\exp(x_1) & 0 & \exp(2x_1)/2& 0\\
0 & 0 & 0 & -1\\
\end{array} \right)
\label{gmatriz}
\end{equation}

donde las coordenadas del espacio tiempo son $x_\mu$, $\mu=
0,1,2,3$, $a$ es un parámetro constante relacionado con la velocidad angular $\Omega$ de la materia respecto al compas de inercia ($a=1/\sqrt{2}\Omega$).

Esta métrica resuelve las ecuaciones de Einstein:
\begin{equation}
R_{ab}+(\Lambda - \frac{1}{2}R)g_{ab}=-\sqrt{8\pi G}~T_{ab},
\label{eqe}
\end{equation}
con constante cosmológica $\Lambda=-1/(2a^2)$ y la fuente de materia está descrita por el tensor de energía momento de un fluido perfecto que consiste de polvo con densidad constante $\rho$
\begin{equation}
T^{ab}=\rho~ u^a u^b
\end{equation}

con $u^a=dx^a/d\tau$ la -cuadrivelocidad- de la materia.

Como sabemos, a partir de la métrica podemos calcular el valor de otras -cantidades-%Porque es importante añadir esta información, saber estos valores?
, por ejmplo, %Se comprobó utilizando grTensor de Maple que 
 los símbolos de Christoffel diferente de cero son:
\begin{eqnarray}
 \Gamma_{012}&=&\Gamma_{120}=\Gamma_{210}= a^2 \exp(x_1)/2, \nonumber
 \\
\Gamma_{122}&=&-\Gamma_{221}=\Gamma_{212}= a^2 \exp(2x_1)/2, \nonumber
\\
\Gamma_{01}^{0} &=& 1, \qquad \Gamma_{22}^{1} = \exp(2x_1)/2, \nonumber
\\
\Gamma_{10}^{2} &=& -\exp(-x_1), \qquad
\Gamma_{12}^{0} = {\Gamma_{02}}^{1} = \exp(x_1)/2, \nonumber
\label{christoffel}
\end{eqnarray}
%los cuales concuerdan con los dados en \cite{godel}

%explicación de la forma de calcular las componentes segun Gödel
Las componentes del tensor de Ricci diferentes de cero son:
\begin{equation}\label{ricci}
R_{00}=1, \quad R_{22}= \exp(2x_1),\quad R_{02}=R_{20}=\exp(x_1),
\label{ricci}
\end{equation}
y el escalar de curvatura
\begin{equation}
R=\frac{1}{a^2},
\label{R}
\end{equation}

Ahora, utilizando las componentes de la métrica (\ref{gmatriz}) y los símbolos de Christoffel (\ref{christoffel}) en la ecuación de las geodésicas
\begin{equation}
\frac {d^2 x_{\alpha}}{d\tau^2} + \Gamma_{\alpha}^{\beta\sigma} \frac{dx_\beta}{d\tau} \frac{dx_\sigma}{d\tau} = 0,
\label{geodesicas}
\end{equation}
donde las derivadas son respecto al tiempo propio $\tau$.

% AGREGAR EXPLICACIONES/COMENTARIOS EN LOS PASOS PARA COMPLEMENTAR LA INFORMACIÓN

Las ecuaciones de geodésicas resultantes %poner un poco de explicación de cómo calcular una de estas ecuaciones
\begin{equation}
\ddot x_0 + 2 \dot x_0 \dot x_1 + \exp(x_1) \dot x_1 \dot x_2 = 0,
\label{eG0}
\end{equation}

\begin{equation}
\ddot x_1 + \exp(x_1) \dot x_0 \dot x_2 + \exp(2 x_1 ) (\dot x_2)^2/2 = 0,
\label{eG1}
\end{equation}

\begin{equation}
\ddot x_2 - 2 \exp(x_1) \dot x_0 \dot x_1 = 0,
\label{eG2}
\end{equation}
%
y
%
\begin{equation}
\ddot x_3 = 0,
\label{eG3}
\end{equation}
%

%\begin{eqnarray}
%&&\label{g0} \ddot x_0 + 2 \dot x_0 \dot x_1 + \exp(x_1) \dot x_1 \dot x_2 = 0,\\ 
%&&\label{g1} \ddot x_1 + \exp(x_1) \dot x_0 \dot x_2 + \exp(2 x_1 ) (\dot x_2)^2/2 = 0,\\ \label{g1}
%&&\label{g2} \ddot x_2 - 2 \exp(x_1) \dot x_0 \dot x_1 = 0,\\
%&&\label{g3} \ddot x_3 = 0,
%\end{eqnarray}
%dfcgvhb \ref{g0}, \ref{g3}, \ref{g1}, \ref{g2}

Solución de la ecuación (\ref{eG3}) es inmediata
\begin{equation}
\dot{x_3}=C \qquad y \qquad x_3=C\tau + c_3
\label{g3}
\end{equation}

donde $C$ y $c_3$ son constantes. % REVISAR los valores en el artículo de rev. mex.

% REVISAR derivación de esta ecuación?
La primera ecuación de las geodésicas es:
\begin{equation}
\left( \dot x_0 + \exp (x_1) \dot x_2 \right)^2 - (\dot x_1)^2 - \exp (2 x_1) (\dot x_2)^2 / 2 - (\dot x_3)^2 = 1.
\label{int1}
\end{equation}

Sumando el producto de la ecuación (\ref{eG1}) por $\dot{x_1}$ y el producto de (\ref{eG2}) por $\exp(2x_1)\dot{x_2}/2$
\begin{equation}
\dot x_1 \ddot x_1 + \frac {1}{2} \exp (2 x_1) \dot x_1 (\dot x_2)^2 + \frac {1}{2} \exp (2 x_1) \dot x_2 \ddot x_2 = 0,
\label{inter1}
\end{equation}
% AGREGAR algún paso de la integración
e integrando se obtiene:
\begin{equation}
B^2=(\dot x_1)^2 + \frac{1}{2} \exp (2 x_1) (\dot x_2)^2,
\label{inter2}
\end{equation}

A partir de (\ref{inter2}) obtenemos $\dot{x_2}$ en función de $\dot{x_1}$ siendo además $B$ una constante
\begin{equation}
\dot x_2 = \sqrt {2} \exp (- x_1) [B^2 - (\dot x_1)^2]^{1/2}.
\label{vel2}
\end{equation}

Utilizando las ecuaciones (\ref{g3}) y (\ref{inter2}) en (\ref{int1})
\begin{equation}
\sqrt{1+B^2+C^2}=\dot{x_0}+ \exp(x_1)\dot{x_2},
\end{equation}
y definimos la constante $D$ como:
\begin{equation}
D \equiv \sqrt{2 (1 + B^2 + C^2) } = \sqrt{2 }\; (\dot x_0 + \exp(x_1)\dot x_2) .
\end{equation}

Esta definición es útil por que al despejar $x_0$ y utilizando el valor de $x_2$ de (\ref{vel2}) se obtiene
\begin{eqnarray}
\dot{x_0}&=&(D/\sqrt{2})- \exp(x_1)\dot{x_2}\nonumber
\\ 
&=&\frac{1}{\sqrt{2}}\left( D-2 \left[B^2-(\dot{x_1})^2 \right]^{\frac{1}{2}} \right) .
\label{vel0}
\end{eqnarray}

Ahora, siguiendo el -procedimiento- de Chandrasekhar y Wright %[\cite{chandra}]
se introduce una nueva variable $\theta$ de forma que
\begin{equation}
\dot{x_1} \equiv B \sin \theta .
\label{vel1}
\end{equation}

Utilizando esta definición, la ecuación (\ref{vel2}) y (\ref{vel0}) en (\ref{eG1})
\begin{eqnarray}
-B \cos \theta \frac{d\theta}{d\tau}&=& \exp(x_1) \left[ \frac{1}{\sqrt{2}} \left( D -2B \cos \theta \right) \right] \left[ \sqrt{2}\exp (-x_1)B \cos \theta \right] \nonumber \\ \nonumber
&+& \frac{1}{2} \exp(2x_1) \left[ 2\exp (-2x_1)B^2 \cos^2 \theta \right]
\end{eqnarray}
%
\begin{equation}
\frac{d\theta}{d\tau}= D - B \cos \theta.
\end{equation}
%como y porque cambiar s por tau
Al integrar esta última ecuación se obtiene una relación entre $\tau$ y $\theta$ de la siguiente forma:
\begin{eqnarray}
\tau &=& - \sqrt{ \frac{4}{(D^2-B^2)}} \arctan \left[ \left( \frac{D+B}{D-B} \right)^{1/2} \tan \left( \frac{\theta}{2} \right) \right] \cr
&\equiv& \frac{2}{(D^2-B^2)^{1/2}}\, \sigma,
\label{sigma}
\end{eqnarray}
en donde se introdujo una nueva variable $\sigma$, la cual se usará de ahora en adelante como el nuevo parámetro de tiempo propio ya que solo difiere de $\tau$ por una constante. Se asume $D>B$, y de la ecuación (\ref{sigma}) se obtiene la relación de $\sigma$ y $\theta$
%AGREGAR las restricciones en los valores de las constantes B, C, D
\begin{equation} 
\tan \sigma = - \sqrt{\frac{D+B}{D-B}} \tan \left(\frac{\theta}{2}\right).
\label{tansigma}
\end{equation}

De esta ecuación y utilizando la siguiente identidad trigonométrica $\tan \left(\frac{x}{2} \right) = \sqrt{\frac{1- \cos x}{1+\cos x}}$, resultan las siguientes relaciones
\begin{equation}
\cos \theta = \frac {1 - \alpha \tan^2 \sigma}{1 + \alpha \tan^2 \sigma}, \quad
\sin\theta = -2 \sqrt{\alpha} \frac{\tan{\sigma}}{ 1+ \alpha \tan^2 \sigma}, 
\hbox{ donde   } \alpha \equiv \frac{D-B}{D+B},
\label{cossen}
\end{equation}

Como en la ecuación (\ref{vel1}) se tiene una relación entre la -cuadrivelocidad- de la coordenada $\mu=1$ y la variable $\theta$ que se acaba de demostrar está relacionada con $\sigma$ la cual se definió como el nuevo parámetro de tiempo propio, se pueden utilizar (\ref{vel1}), (\ref{sigma}), (\ref{tansigma}) y (\ref{cossen}) para escribir lo siguiente
% elemento de línea s o tau?
\begin{equation}
\frac{dx_1}{d\sigma}=\frac{dx_1}{d\tau}\frac{d\tau}{d\sigma}=\dot{x_1}\frac{d\tau}{d\sigma}=- 2B \sqrt{\frac{D-B}{D+B}}\; \frac{\tan \sigma}{1+\alpha \tan^2 \sigma} \left[ \frac{2}{(D^2-B^2)^{1/2}} \right]
\end{equation} 
%
\begin{equation}
\frac{dx_1}{d\sigma}=- \left( \frac{4B}{D+B} \right) \frac{\tan \sigma}{1+\alpha \tan^2 \sigma}.
\label{dx1}
\end{equation}

De la misma forma para %geodésicas de primer orden
(\ref{vel2}), (\ref{vel0}) y (\ref{g3})
%
\begin{equation}
\frac{dx_{0}}{d\sigma}=\left(\frac{2}{D^{2}-B^{2}}\right)^{1/2}\left[D-2B\left(\frac{1-\alpha \tan^{2}\sigma}{1+\alpha \tan^{2}\sigma}\right)\right],
\label{dx0}
\end{equation}
%
\begin{equation}
\frac{dx_{2}}{d\sigma}=2e^{-x_1}\left( \frac{2B^2}{D^2-B^2} \right)^{1/2} \frac{1-\alpha \tan^{2}\sigma}{1+\alpha \tan^{2}\sigma},
\label{dx2.1}
\end{equation}

\begin{equation}
\frac{dx_{3}}{d \sigma}= \frac{2C}{(D^2-B^2)^{1/2}}.
\label{dx3}
\end{equation}

De esta forma se desacoplaron las ecuaciones de velocidad para todas los valores de $\mu$ de forma que quedaron solo en función del tiempo propio $\sigma$.

Ahora solo resta resolver estas cuatro ecuaciones diferenciales, comenzando para la coordenada temporal
%\begin{equation}
\[
\int dx_0 = \int \left(\frac{2}{D^{2}-B^{2}}\right)^{1/2}\left[D-2B\left(\frac{1-\alpha \tan^{2}\sigma}{1+\alpha \tan^{2}\sigma}\right)\right] d\sigma,
\]
%\end{equation}
% ESCRIBIR un poco del desarrollo de la integral

\begin{equation} 
x_{0}=-\left(\frac{2D^{2}}{D^{2}-B^{2}}\right)^{1/2}\sigma+2\sqrt{2}\arctan(\sqrt{\alpha}\tan\sigma)+c_{0}. 
\label{x0}
\end{equation}

-Para la primera coordenada espacial-:
\[
\int dx_1 =- \int \left( \frac{4B}{D+B} \right) \frac{\tan \sigma}{1+\alpha \tan^2 \sigma} \; d\sigma,
\]
%
\begin{equation}
x_{1}=\log\left(\frac{1+\alpha \tan^{2}\sigma}{1+\tan^{2}\sigma}\right)+c_{1},
\label{x1}
\end{equation}

Ahora, aunque la ecuación (\ref{dx2.1}) aún tiene un término en función de $x_1$, eso se -arregla- fácilmente al sustituir el valor de (\ref{dx1}) en esta ecuación
\begin{eqnarray}
\frac{dx_{2}}{d\sigma}&=&2 \exp \left( -\log\left[\frac{1+\alpha \tan^{2}\sigma}{1+\tan^{2}\sigma}\right] -c_{1} \right) \left( \frac{2B^2}{D^2-B^2} \right)^{1/2} \frac{1-\alpha \tan^{2}\sigma}{1+\alpha \tan^{2}\sigma}, \nonumber
\\
\nonumber
&=&2 e^{-c_1}\left( \frac{2B^2}{D^2-B^2} \right)^{1/2} \left( \frac{1+\tan^{2}\sigma}{1+\alpha \tan^{2}\sigma} \right) \left( \frac{1-\alpha \tan^{2}\sigma}{1+\alpha \tan^{2}\sigma} \right),
\end{eqnarray}

\begin{equation}
\frac{dx_{2}}{d\sigma}=2e^{-c_{1}}\left(\frac{2B^{2}}{D^{2}-B^{2}}\right)^{1/2}\frac{(1-\alpha \tan^{2}\sigma)\sec^{2}\sigma}{(1+\alpha \tan^{2}\sigma)^{2}}.
\label{dx2}
\end{equation}

Integrando esta ecuación y (\ref{dx3}) se obtienen
\begin{equation}
x_{2}=2e^{-c_{1}}\left(\frac{2B^{2}}{D^{2}-B^{2}}\right)^{1/2} \frac{\tan\sigma}{1+\alpha \tan^{2}\sigma}+c_{2},
\label{x2}
\end{equation}
% TALVEZ escribir la integración de esta ecuación x2
\begin{equation}
x_{3}=\frac{2 C}{(D^2-B^2)^{1/2}}\, \sigma + c_3.
\label{x3}
\end{equation}

% PREGUNTA, ¡tiene alguna importancia que para mu=0,1,2 las geodésicas sean periódicas y para mu=3 sea lineal?

Estas cuatro ecuaciones son las soluciones a las ecuaciones de las geodésicas en el modelo de universo de Gödel, si se grafican tomando las siguientes consideraciones: $c_0=c_1=c_2=c_3=0$, $\alpha = 1/4$ y $C= 1/3$ % justificar los valores
se obtienen las curvas mostradas en la figura (\ref{geodesicas}), donde la discontinuidad de la coordenada temporal $x_0$ es una consecuencia de la elección de las coordenadas y no tiene relación a la estructura del espacio tiempo.

Cabe mencionar que las soluciones que se encontraron son las mismas que las de Chandrasekhar y Wrigh en \cite{chandra}, la única diferencia significativa con su trabajo es en la gráfica de la coordenada $x_0$ que a pesar de ser la misma ecuación, ellos presentan una curva con un máximo en $\sigma=3\pi /4$ mientras que en la figura (\ref{geodesicas}) la curva no tiene ningún máximo y tiene una discontinuidad en $\sigma =\pi/2$.

% VOLVER A HACER LA GRÁFICA
\begin{figure}[h!]
\centering
\includegraphics[height=3in]{grageo}
\caption{Solución de las ecuaciones de las geodésicas.% más explicación
 }
\label{geodesicas}
\end{figure}

Como ya se comentó en el capítulo anterior, una de las características de este modelo de universo es que la materia en él se encuentra en rotación como un cuerpo rígido con respecto al compás de inercia, por eso podemos hablar de simetría cilíndrica. 
% REVISAR, ¿por qué no esférica? o cualquier otra simetría

Por -conveniencia- y para hacer explícita esta simetría %que seguro ya se hizo en el capítulo anterior...
en las soluciones, se hará un cambio de coordenadas a ($r, \varphi, t, z$) haciendo uso de las transformaciones que Gödel -plantea- \cite{godel}. 
% REVISAR, derivación de estas transformaciones

\begin{equation}
e^{x_{1}}=\cosh 2r+cos\varphi \sinh 2r,
\label{tx1}
\end{equation}
%
\begin{equation}
x_2 \, e^{x_{1}}=\sqrt{2}\sin\varphi \sinh 2r,
\label{tx2}
\end{equation}
%
\begin{equation}
\tan\left(\frac{\varphi}{2}+\frac{x_0-2t}{2\sqrt{2}}\right)=e^{-2r}\tan\frac{\varphi}{2},
\label{tx0}
\end{equation}
%
\begin{equation}\label{13}
x_{3}=2z.
\label{tx3}
\end{equation}

En estas nuevas coordenadas la métrica (\ref{elinea}) se transforma -a-:
\begin{equation}
ds^{2}=4a^{2}(dt^{2}-dr^{2}-dz^{2}+(\sinh^{4}r-\sinh^{2}r)d\varphi^{2}+2^{3/2}\sinh^{2}r d\varphi dt);
\label{elineac}
\end{equation}
% REVISAR lo siguiente
en esta nueva forma de la métrica se vuelve explícita la simetría cilíndrica ya que no tiene dependecia en $\varphi$.

% AGREGAR más explicación, basado en chandrasekhar, en la siguiente sección 


%Explicar un poco de las geodésicas nulas que también resultan para cierto valor de cosh 2r


 \begin{thebibliography}{99}
 %orden alfabético o por aparición
 \bibitem{wwt} Yourgrau, P.. (2005). \emph{A World Without Time}. New York %(?)
 : Basic Books.
 
 \bibitem{remark} Gödel, K.. (1949). A Remark about the Relationship Between Relativity Theory and Idealistic Philosophy. \emph{P. A. Schilpp},pp. 557-562
 
 \bibitem{godel} An Example

 \bibitem{chandra}
 
 \bibitem{inverno} d'Inverno, R.. (1992). \emph{Introducing Einstein's Relativity}. 
 
 \bibitem{rindler} Rindler, W.. ( junio 2009). Gödel, Einstein, Mach, Gamow, and Lanczos: Gödel’s remarkable excursion into cosmology. \emph{Am. J. Phys}, 77, 498-510.
 
 
 \end{thebibliography}
 
\end{document}00